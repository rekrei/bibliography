\subsection{Project Mosul: Preserving the Past Through Crowd-Sourced Imagery}

\textbf{Authors:} \index{Vincent, Matthew L.}Matthew L. Vincent, \index{Coughenour, Chance} Chance Coughenour, \index{Remondino, Fabio} Fabio Remondino
\\
\textbf{Abstract:} Project Mosul is a project that grew out of a reaction to the mass destruction of cultural heritage, particularly that of The Cultural Museum of Mosul. Using crowd-sourced imagery from tourists and professionals alike, lost heritage has been virtually reconstructed using semi-automatic photogrammetric processes. While these virtual reconstructions serve to preserve the memory of the lost heritage, they cannot replace that which has been lost. The project embraces crowd-sourced principles, and strives to involve the volunteer community in every aspect of the project. The models resulting from the photogrammetric process are open and free to the public to access; however, these models have limited scientific application due to the inability to quantify the geometric fidelity. That being said, the ability to visually preserve and present the heritage through virtual museums and exhibits. Future work will involve mixed media exhibitions incorporating both physical reproductions and virtual visualisation tools helping to connect the public with the heritage and raise awareness of the value of our global heritage. Beyond the representations and exhibitions, we are working towards the integration of documentation of the heritage in order to retell the complete story of that which has been lost..

\index{Project Mosul}\index{poster}
