\subsection{Project Mosul - Preserving the Memory of Lost Heritage}

\textbf{Authors:} \index{Coughenour, Chance} Chance Coughenour, \index{Vincent, Matthew L.}Matthew L. Vincent
\\
\textbf{Conference}: Conference on Cultural Heritage and New Technologies, Vienna, Austria, 2-4 November, 2015
\\
\textbf{Abstract:} The principal objective of Project Mosul is to foster a volunteer initiative which recreates virtual models of recently destroyed cultural heritage from crowd-sourced data. Although originally launched following the release of the shocking video of devastation at the Mosul Museum in Iraq, the project has expanded its attention to the nearby sites of Nimrud and Hatra. Unfortunately, cultural heritage in danger is not only limited to man-made destruction, as demonstrated by the recent earthquake in Nepal. So too, the project seeks to widen its aim in utilizing photogrammetry to help preserve the memory of this valuable heritage for future generations and empower anyone interested to participate. Furthermore, the website’s tools have been developed with an emphasis on open source, thereby allowing users to take part in their development.
The results thus far are compelling where some artifacts have been reconstructed from less than ten images. Of course, as the number of acquired photos increases, so will the amount and quality of 3D models of these lost artifacts and archaeological structures. Thanks to the integrated 3D gallery on the website, volunteers’ work can instantly be visualized. Although from a scientifically-grounded perspective, it may never be possible to verify the precision of each virtual reconstruction since the original is now gone. At least its virtual substitute helps to preserve the memory of what was lost.
By expanding on previous investigations and technology, Project Mosul is a concept that was bound to materialize, even outside of official heritage organizations. Its innovative concept has attracted the attention of people and organizations from the public and private sectors alike. Our hope is to not only to point out the growing need to protect cultural heritage in danger but we also hope to present one example of how this may be achieved.

\index{Project Mosul}\index{conference presentation}
