\subsection{Crowd-sourcing the 3D digital reconstructions of lost cultural heritage}

\textbf{Authors:} \index{Vincent, Matthew L.}Matthew L. Vincent, \index{Coughenour, Chance} Chance Coughenour, \index{Remondino, Fabio} Fabio Remondino, \index{Flores Gutierrez, Mariano} Mariano Flores Gutierrez, \index{Lopez-Menchero Bendicho, Victor Manuel} Victor Manuel Lopez-Menchero Bendicho, \index{Fritsch, Dieter} Dieter Fritsch
\\
\textbf{Conference}: Digital Heritage 2015, Granada, Spain, 28 September - 2 October, 2015
\\
\textbf{Abstract:} Crowd-sourced photogrammetric reconstructions offer a unique opportunity for the digital visualisation of lost heritage. Project Mosul is a project that seeks to digitally reconstruct lost heritage, whether through war, conflict, natural disaster or other means, and preserve the memory of that heritage through digital preservation schemes. The project is not without its challenges, however. For example, geometric fidelity is impossible to determine, maintaining community interest, while returning some value to the research community. That being said, textured 3D models can still be a valuable source for visualization, memory and documentation.

\index{Project Mosul}\index{history}\index{cultural aspects}\index{data visualisation}\index{history}\index{image reconstruction}\index{3D digital reconstructions}\index{crowd-sourced photogrammetric reconstructions}\index{crowd-sourcing}\index{digital preservation schemes}\index{digital visualisation}\index{documentation}\index{geometric fidelity}\index{lost cultural heritage}\index{project Mosul}\index{research community}\index{textured 3D models}\index{visualization}\index{Cameras}\index{Cultural differences}\index{Earthquakes}\index{Image reconstruction}\index{Solid modeling}\index{Three-dimensional displays}\index{Visualization}\index{Crowd-sourced}\index{citizen science}\index{photogrammetry}